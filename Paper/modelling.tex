\section{Power Modelling}
\label{sec:modelling}

We consider two related issues in the practical component of our work. The first task is to establish bounds on how much CPU power draw can vary whilst running arbitrary code. This allows us to place an upper limit on how much is possible to optimize a given code by. These bounds also help us tackle a second issue - namely to provide an objective appraisal of the power models proposed elsewhere in the literature.

The only relevant property an engineer can influence when optimizing software for reduced power consumption is the activity factor. As previously discussed, the activity factor of a processor has a linear relationship with dynamic power and secondary non-linear relationships through the actions of DVFS with both static and dynamic power. A developer wishing to save energy should therefore aim to modify their code to reduce its contribution to activity factor.

Although activity factor is defined as a scalar between zero and one, we know that in practice its range will be much more limited. Although some processor elements are responsible for handing program logic, many more are dedicated to ongoing ancillary tasks like instruction decoding or propagating clock signals. Conversely, we know that processors are not able to keep all their functional units active simultaneously.  We define the range of values activity factors can reasonably take whilst running a code as $[\alpha  .. \beta]$ where $0 < \alpha < \beta < 1$.

\begin{figure}[ht]                                                               
\centering                                                                      
\lstset{basicstyle=\ttfamily\footnotesize\bfseries,                             
      frame=tb}                                                                 
\lstinputlisting[]{Listings/nops.c}                             
\caption{Baseline Power Benchmark}                            
\label{fig:nopjmpres}                                                           
\end{figure}  

We rely on empirical measurement to discover the power draws $P_{\alpha}$ and $P_{\beta}$ associated with activity factors $\alpha$ and $\beta$. Our first benchmark, shown in \figurename~\ref{fig:nopjmpres}, is a simple busy waiting loop with no side-effects. By monitoring the power consumed whilst running this code we derive our lower bound ${P_{\alpha}}$. It is safe to assume that any non-trivial code requires strictly more power than our benchmark which performs no work. We repeat this process to produce a baseline for each combination of clock frequency and core count as shown in Table~\ref{tab:baseline}.

Having established our $P_{\alpha}$ baseline we now turn our attention to the $P_{\beta}$ limit. There is no obvious way to construct an artificial benchmark which exercises a maximal set of logic elements. Fortunately, when optimizing a specific code there is no need to devise such a system-wide upper bound. We simply use the power draw of our  un-optimized code as our upper limit, because trivially a worthwhile code optimization cannot yield worse performance.

We plot lines with gradients $P_{\alpha}$ and $P_{\beta}$ in \figurename~\ref{fig:pow} to establish a feasible performance envelope. This represents the set of all $(P_{opt}, t_{opt})$ pairs for which $P_{\alpha} < P_{opt} < P_{\beta}$, with no bounds placed on runtime, $t_{opt}$. It should be clear that any optimisation must exist somewhere within this envelope.

To constrain our search further we must consider the metric we wish to reduce. We know that for two logically equivalent codes $a$ and $b$, the transformation $a \to b$ is a valid optimization with respect to a cost metric $M$ if and only if $M(a) > M(b)$. In other words, if we plot a linking all points $a$ in our envelope where $M(a) = M(unopt.)$ then any valid optimization can only exist below this line.

Our final bound concerns what it means to optimize for energy efficiency. All else being equal, energy to solution will reduce if a code can be made to run faster. That said, it would be disingenuous to claim this inevitable side-effect of classical optimization is also a valid energy optimization. The simplest solution would be to state that true power optimizations do not reduce runtime. We feel this definition is overly restrictive, however, as it excludes hypothetical optimizations which deliver significant reductions in power draw with minuscule reductions in runtime.

The definition we have settled on is that the transformation $a \to b$ is a valid power optimization with respect to a chosen metric if the reduction in power draw it delivers accounts for the greatest proportion of improvement in said metric. Formally, we say \todo{Tricky equation}. Above this bound the reduction in runtime contributes a larger improvement to the cost metric than any reduction in power, meaning code changes in this region should be considered as classical runtime optimizations. By extension, all valid power optimizations must lie below this bound.


\begin{figure}
\includegraphics[width=0.9\linewidth]{./Plots/edp_optimization_space/optspace-figure0.pdf}
\caption{Power Optimization Window}\label{fig:pow}
\end{figure}

In \figurename~\ref{fig:pow} we plot the limits described above for a hypothetical code. 





We first construct a power model in the vein of those proposed in the literature. We then try to link the predictions of our model to the power equations given previously. Essentially we are attempting to test against the null hypothesis - to show how much better or worse our regressed model is better than a naive attempt.


\todo{Make sure table is in the right place}





\fragment{We feel it is disingenuous to speed up a code by a large factor and then claim to have optimized for energy. }

\todo{the definitions matter}

\fragment{ 
The latter proves particularly onerous, as by definition the circuitry driven by a clock signal has an activity factor of 1 as it changes state each cycle \findref. 
}





\todo{Occam's razor}

\fragment{Often power models are assessed by their performance relative to a measured baseline. Although this is important, this figure is somewhat meaningless without context. Our approach is one of quantifying 


\todo{Make this bit less attacky:}
It is also worth noting that this model is effectively useless from a code optimization standpoint as it does not take any software features into account. This property is intentional, as it allows us once again to provide a baseline from which to assess any models. One can only realistically expect to optimize a code to the level at which any changes can be accurately measured. A model which claims to assist in the optimization of codes can only offer optimizations to the extent it shows divergence from this baseline.

}
\fragment{The unknown quantities in our simplified power equations can be empirically measured.}
Baseline against which to compare the work of others. 


\fragment{Accuracy figures without context are notoriously unreliable. To compensate for this we compare the outputs of various models against a baseline we have devised. This baseline consists of what we regard as the simplest non-trivial power model conceivable. This model stands in as a sort of null hypothesis test, our justification being that a complex model only adds value to the extent with which it outperforms this toy model.}

\fragment{Our toy model is not the simplest model possible - It is well established and readily apparent that runtime is the largest contributory factor to power consumption. One could therefore imagine a simple power model}

\fragment{We consider this to be the absolute minimum power consumption possible.}

\fragment{Intentionally simplistic. Our decision to simplify activity factor to active cores is part of this. We assume that instruction pipe-lining does a reasonable job of keeping as much silicon active as possible, and we do not know how much area an individual instruction activates, and a large percentage of power use is from clock circuitry anyway}



\fragment{Two components to our investigation. Firstly, the upper bound imposed by the baseline power consumption. Secondly, as we can only view power figures approximately, the error introduced into these models necessarily limits their usefulness as optimization tools beyond a certain point.}



\begin{table}


\centering
\small
\begin{tabular}{@{}ccccc@{}} \toprule
&\multicolumn{4}{c}{CPU Cores Active} \\ \cmidrule(r){2-5}
Frequency (GHz) & 1 & 2 & 3 & 4 \\ \midrule 
1.60 & 9.180 & 10.970 & 12.832 & 14.555 \\ 
1.70 & 9.449 & 11.446 & 13.295 & 15.112 \\ 
1.80 & 9.592 & 11.654 & 13.617 & 15.682 \\ 
1.90 & 9.816 & 12.009 & 14.168 & 16.291 \\ 
2.10 & 10.272 & 12.709 & 15.161 & 17.605 \\ 
2.20 & 10.559 & 13.161 & 15.705 & 18.333 \\ 
2.30 & 10.812 & 13.551 & 16.419 & 19.070 \\ 
2.40 & 11.303 & 14.290 & 17.012 & 19.946 \\ 
2.50 & 11.680 & 14.784 & 18.000 & 20.837 \\ 
2.60 & 11.819 & 15.144 & 18.616 & 21.879 \\ 
2.70 & 12.205 & 15.830 & 19.379 & 22.940 \\ 
2.90 & 13.095 & 17.196 & 21.155 & 25.344 \\ 
3.00 & 13.547 & 18.160 & 22.210 & 26.759 \\ 
3.10 & 14.048 & 18.870 & 23.639 & 28.284 \\ 
3.20 & 14.504 & 19.726 & 24.940 & 29.857 \\ 
\bottomrule
\end{tabular}
   \vspace{0.5\baselineskip}
\caption{Test Platform Base CPU Power (W)}
\label{tab:baseline}
\end{table} 



\todo{table 2 - benchmark results. Linpack is 35.2428 for 100 seconds}
\todo{Something about how any delta is about how the code uses more logic elements than our NOP lo ops. Our optimization window is therefore within that Delta}

\todo{Be nice, say that this does not invalidate other work, but simply shows that hardware has now reached \reword{convergence} and basically there's little traction left}

Having shown that 

\fragment{This model is not supposed to be rigorous or precise. Rather we present it as the simplest possible non-trivial power model which accounts for the sources of variability in the power equations. In effect our model is functionally equivalent to the power equations
presented previously with appropriate constants substituted \todo{sampled}. We present this model as our null hypothesis - for a model to be useful it must outperform this one. It is necessarily an oversimplification - it ignores clock gating. It should also consistently underestimate the true value, as the benchmark selected intentionally exercises the minimum possible number of logic elements while still performing work.}


\todo{One limitation of this work stems from the nature of RAPL - it is an accumulative measurement of power taking into account all processes. On a loaded system it will measure all tasks }

\begin{figure}
\label{fig:dummy_model}
\includegraphics[width=0.9\linewidth]{./Plots/dummy_model/dummymodel-figure0.pdf}
\caption{Feasible Power Envelope}
\end{figure}

\todo{Decompose the model - find baseline vs non-baseline components}
\todo{This kind of shows us that the baseline dominates}

\fragment{Limitations to optimizations - the superfluous and the logical equivalences. The first is a no-brainer and boils down to removing unnecessary pre-fetching}

\fragment{Either optimizations which are off the critical path, or else those which are on the critical path }

\fragment{Put another way, any optimization which trades runtime for power has a limited window of}

\todo{equationify fact that energy is power * time, and assume we have power decreases, time static or increases}
\todo{equationify baseline lt optimized lt unoptimized lt roofline} \todo{note - roofline is tdp max}

\todo{Do maths think - by what margin would power have to go down to justify longer runtime? ratio of cost per watt, amortized cost per second}

Nothing discussed so far precludes power optimization in practice.
\todo{imply limits thus far are theoretical}
Even these tight limits may still admit some benefits at extreme scale.
Our final argument however is strictly economic.
\reword{A great deal of attention is paid to the fact that power costs are approaching parity with machine construction costs.} The \choice{implicit, unspoken} \choice{consequence, corollary, implication} being that this has not yet happened. \todo{Ultimate point being here the price difference, machine vs power cost places a further limit on optimization utility. Even if we manage to find a slower, more power efficient method of computing a given result, the cost of energy saved has to be less than the added amortized runtime cost.}


\todo{Legitimate targets for optimization: removing redundant prefetch operations as per phi paper.}
